\documentclass{article}
\usepackage{./paper}

\title{Digital Camera Image Processing Pipeline Components}
\author{Hadleigh Nunes and Evan Lloyd New-Schmidt}
\date{Fall 2018}

\begin{document}

\maketitle

\begin{abstract}
    In this paper we explore concepts and methods for processing digital images in the initial stages of image development, including demosaicing, gamma correction, and white balance correction.
\end{abstract}

\section{Introduction}


% TODO: fig: overview of pipeline components

% TODO: fig: bayer grid image
\cite{picameradocs}

% TODO: fig: bayer images comparison ala wikipedia

% TODO: fig: gamma curves
% TODO: fig: gamma correction image comparison

% TODO: fig: white balance comparison

\section{Mathematical Background}
The mathematical procedures we use are based mainly in Linear Algebra, Multivariable Calculus and Signal Processing. 
Convolution is a process which 

\begin{itemize}
    \item background from linear algebra/ multivariable calculus
    Variance, covariance, eigenvectors, PCA, Gradient, level curves 
    
    \item background from signal processing, specifically digital signal processing.
    Convolution, convolution in two dimensions. 
    
\end{itemize}

\section{Quantitative Procedure}

\subsection{Debayering}
    The simplest method of debayering is nearest neighbor interpolation, which we implemented by convolving a  $3x3$ kernel with each of the color channels. In the raw image, the matrix of each color channel contains values only in pixels corresponding to a sencel of the corresponding color. Pixel locations corresponding to a different color channel are occupied by zeros. By convolving the matrix 
    \begin{align}
        \threebythree{1}{1}{1}{1}{1}{1}{1}{1}{1}
    \end{align}
    with each channel we are essentially averaging the adjacent pixel values of each pixel, including those diagonally adjacent. This is a rudimentary technique, and will create some artifacts, one of the more prominent being a 'zippering' effect caused by %TODO: details on zippering
    
    There are many more complex approaches to debayering designed to reduce the artifacts created in the process. We implement 
\subsection{Gamma Correction}
    Raw digital images are often hard to interpret, as even when scaled to a full RGB range, the non-white parts of the image appear dark with little contrast. This effect occurs because the human eye perceives light non-linearly, meaning that though there may be a significant linear numerical variance in darker regions of an image, those differences are hard to interpret visually. To correct for this, the pixel values of a digital image are adjusted according to a power function. 
    \begin{equation}
        V_{out} = AV_{in}^{\gamma} \label{eqn:Gamma}
    \end{equation}
    The expression \eqref{eqn:Gamma} is the general form of the gamma function, where $A$ is a constant scalar, $V_{in}$ and $V_{out}$ are the input and output signals, respectively. 
    Typically for a "decoding" $\gamma$, a $\gamma$  function meant to re-scale a digital image for human viewing, a $\gamma < 1$ is used to enhance the visibility of non-white areas. 
\subsection{Sharpening}
    We use another convolution process to sharpen images. The kernel 
    \begin{align}
        \threebythree{0}{-1}{0}{-1}{5}{-1}{0}{-1}{0}
    \end{align}
    when convolved with an image produces a sharpened image, as color values which are highly present in neighboring pixels are reduced if not as present in the target pixel.  


\subsection{White Balance}

\subsection{Further Applications}


\section{Results and discussion}

\section{Conclusion}


\section{Equations} 
\begin{equation}
    cov = (\textbf{X} - \mu_{x})(\textbf{X} - \mu_{x})^{T}
\end{equation}

\begin{equation}
    \textbf{M} = \textbf{U} \textbf{$\Sigma$} \textbf{V}
\end{equation}

\begin{equation}
    \textbf{U} = (\textbf{X} - \mu_{x}) \textbf{V} \Sigma^{-1}
\end{equation}

\begin{equation}
    \textbf{W} = \textbf{U}_{Basis}^{T} \textbf{Im}
\end{equation}

% REFERENCES
% References are listed in BibTeX format in the `references.bib` file.
% You can use http://www.citationmachine.net/bibtex/ to create citations in that format.
% Google Scholar will also output BibTeX - click on the quotes to cite and then select "BibTeX".
% Zotero: right-click -> export... -> bibtex format
% entries will only show up if they are used with \cite

\bibliographystyle{IEEEtran}
\bibliography{IEEEabrv,references.bib}

\end{document}